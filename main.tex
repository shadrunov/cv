\documentclass[12pt, a4paper]{extarticle}
\usepackage[top=1cm,left=1.5cm,right=1.5cm,bottom=1cm]{geometry} % Modify margins


\usepackage{polyglossia}
\usepackage{graphicx}
\usepackage{url} % поддержка url-ссылок
\usepackage{hyperref} % пакет для интеграции гиперссылок
\usepackage{multirow}
% \usepackage{arrow}
\usepackage{setspace}
\usepackage{titlesec}

% \usepackage[none]{hyphenat}
\tolerance=1
\emergencystretch=\maxdimen
\hyphenpenalty=10000
\hbadness=10000


\usepackage{enumitem}
\setlist{align=right,topsep=0pt,nolistsep}
% \renewcommand{\labelitemi}{-}


\usepackage{multicol}

\setdefaultlanguage[spelling=modern]{russian}
\setotherlanguage{english}

\usepackage{fontspec}

\newfontfamily{\cyrillicfont}{CMU Serif}[Script=Cyrillic]
\newfontfamily{\cyrillicfontrm}{CMU Serif}[Script=Cyrillic]
\newfontfamily{\cyrillicfontsf}{CMU Serif}[Script=Cyrillic]
\newfontfamily{\cyrillicfonttt}{Courier New}
\defaultfontfeatures{Ligatures=TeX}
\urlstyle{same} % шрифт для URL-ссылок
\linespread{1.5} % междустрочный интервал
\setlength{\parindent}{1.25cm} % отступ для абзаца
% \graphicspath{ {Images} } % Путь до папки с изображениями

% Цвет гиперссылок и цитирования
\usepackage{hyperref}
\hypersetup{colorlinks=true, linkcolor=black, filecolor=blue, citecolor=black, urlcolor=blue}

\usepackage[section]{placeins}
% \captionsetup[figure]{labelformat=simple, labelsep=endash}
% \captionsetup[table]{labelsep=endash, justification=raggedright, singlelinecheck=off}
% \captionsetup[lstlisting]{labelformat=simple, labelsep=endash, font=sf}

% basic font size for headings 


% indentation for items


\pagestyle{empty} % Disable all page numbering
\setlength{\parindent}{0pt} % Stop paragraph indentation
\usepackage[export]{adjustbox}
\begin{document}


% \titleformat{\section}{\bfseries}{\thesection}{1em}{}
% \titleformat{\subsection}{\bfseries}{\thesubsection}{1em}{}
% \titleformat{\subsubsection}{\bfseries}{\thesubsubsection}{1em}{}

% spacing before and after paragraphs 
\titlespacing\section{0pt}{0pt plus 0pt minus 0pt}{0pt plus 0pt minus 0pt}
\titlespacing\subsection{0pt}{0pt plus 0pt minus 0pt}{0pt plus 0pt minus 0pt}
\titlespacing\subsubsection{0pt}{0pt plus 0pt minus 0pt}{0pt plus 0pt minus 0pt}



\setlength{\tabcolsep}{0pt}

\raggedright

% \setlength{\arraystretch}{1pt}

\begin{tabular}{ p{0.7\textwidth} @{\hskip 0.05\textwidth} p{0.25\textwidth} }
    \section*{Алексей Шадрунов}
    \subsection*{DevOps-инженер}

    \setlength{\parindent}{1cm} Студент четвёртого курса НИУ ВШЭ (ИБ). Увлекаюсь DevOps, системным администрированием, виртуализацией, поддержкой инфраструктуры и приложений. 
    
    \setlength{\parindent}{1cm} Предпочитаю работать методично и следовать лучшим практикам. В свободное время пользуюсь Linux, разрабатываю проекты на Python, изучаю языки и помогаю проводить Московский полумарафон. 
    &
    
    \raisebox{-\totalheight}{\includegraphics[width=0.2\columnwidth,center]{maxim.png}}  
    \linebreak 
    \linebreak 
    \small
    \href{mailto:asshadrunov@gmail.com}{asshadrunov@gmail.com} 
    
    \href{https://t.me/asshadrunov}{t.me/asshadrunov} 
    
    \href{https://github.com/shadrunov}{github.com/shadrunov} 
    
    +7 (999) 543-70-29
    % \hline
\end{tabular}

\subsection*{Навыки}
\begin{multicols}{2}
    \begin{itemize}
        \item Ansible, Terraform
        \item CI/CD (GitLab, TeamCity)
        \item Docker, виртуализация (libvirt)
        \item Kubernetes (Rancher, Kubeadm)
        \item Linux, Windows AD, IPMI
        \item Apache Kafka, Mesos, Hadoop
        \item Monitoring (Elastic, Grafana)
        \item Python (Django, SQL, Kafka)
        \item C/C++, PHP, Bash, SQL
        \item Информационная безопасность
    \end{itemize}
    \end{multicols}

\subsection*{Опыт работы}

\begin{tabular}{ p{0.1\textwidth} @{\hskip 10pt} p{0.85\textwidth} }
    \raisebox{-\totalheight}{\includegraphics[width=0.1\columnwidth,right]{kaspersky.png}} 
    &
    {\bf с июля 2022, Лаборатория Касперского}
    
    Стажёр DevOps, команда разработки инфраструктуры

    \begin{itemize}
        \item Настраивал инфраструктуру для CI/CD (TeamCity и build-агенты, сборка образов, интеграционное тестирование, деплой);
        \item Управлял серверами и ВМ с помощью Ansible и RHEL Kickstart;
        \item Разворачивал приложения в Kubernetes, логировал события из Rancher;
        \item Настраивал мониторинг инфраструктуры, аггрегацию метрик;
        \item Восстанавливал работу внутренних сервисов после инцидентов.
    \end{itemize}
    \\
    \raisebox{-\totalheight}{\includegraphics[width=0.1\columnwidth,right]{pt.png}} 
    &
    {\bf февраль — июнь 2022, Positive Technologies}
    
    Младший специалист, отдел SOC
    
    \begin{itemize}
        \item занимался мониторингом и расследованием инцидентов
    \end{itemize}

    % \hline
\end{tabular}


\subsection*{Образование}

\begin{tabular}{ p{0.1\textwidth} @{\hskip 10pt} p{0.85\textwidth} }
    \raisebox{-\totalheight}{\includegraphics[width=0.1\columnwidth,right]{hse.png}} 
    &
    {\bf 2020 — 2024}

    {\bf НИУ ВШЭ}, Московский институт электроники и математики им. Тихонова

    {\it Информационная безопасность (бакалавриат)}
    \\
    % \hline
\end{tabular}



\subsection*{Дополнительное образование}
\small
\begin{itemize}
    \item {\bf \href{https://docs.google.com/document/d/1TC7nxMp90rXYOoCR8xFcSec68fIHsvFjR1trdSvrNIM/edit?usp=sharing}{Программа повышения квалификации "DevSecOps и технологии контейнеризации"}} \\
    МИЭМ НИУ ВШЭ \\
    (2022)
    \item {\bf Курс «Облачные вычисления» (Yandex Cloud)} \\
    НИУ ВШЭ, Факультет компьютерных наук \\
    (2022)
    \item {\bf Летняя школа МИЭМ НИУ ВШЭ «DevOps и CI/CD»} (\href{https://drive.google.com/file/d/1mH_22bA2mgqw43ehw-CvZM6c7_8BLLYm/view?usp=sharing}{drive}) \\
    (2021)
    \item {\bf \href{https://drive.google.com/file/d/1RVzl6slqWSsdnuELYxT9DYE8HfRkd_6O/view?usp=sharing}{Сертификат} специалиста по технологии ViPNet} \\
    Учебный центр ИнфоТеКС \\
    (2021)
    \item {\bf \href{https://drive.google.com/file/d/1F6PAzi25VpJ1SL9G5TVafDHkP2hBAEeB/view}{Сертификат} специалиста по администрированию InfoWatch Traffic Monitor 7} \\
    Infowatch \\
    (2021)
    \item {\bf Курс по СУБД и SQL} \\
    Databases and SQL for Data Science with Python (\href{https://drive.google.com/file/d/1lifQoGFJDSxcjNfwNnR1FiF0MavP9uJL/view?usp=sharing}{drive}) \\
    (2021)
    \item {\bf Курс по сетевому администрированию} \\
    Cisco CCNA: R\&S (\href{https://drive.google.com/file/d/1fBK2Zp-YPH3ahQ8KUEko2ggwCXJ_MYSj/view}{drive}) \\
    (2021)
    \item {\bf Курс по Python} \\
    Python: основы и применение (\href{https://drive.google.com/file/d/1opeiH4ZQKNo-1PUl4RhARUt85vdAHb3L/view?usp=sharing}{drive}) \\
    (2020)
\end{itemize}


\subsection*{Достижения}
\small
\begin{itemize}
    \item {\bf Всероссийская олимпиада студентов «Я — профессионал»} \\
    \href{https://drive.google.com/file/d/1m03jSHaAR2LvP1NXdzZBnVJ_jEoJGz5P/view?usp=sharing}{призёр} в направлении «Безопасность информационных систем и технологий критически важных объектов» \\
    (2022)
    \item {\bf III отраслевой чемпионат Digitalskills 2021} \\
    \href{https://drive.google.com/file/d/1ghvxSiuflWyP6pI33bKbf_N1M8ZIhiTY/view?usp=sharing}{диплом 1 степени} в компетенции «Корпоративная защита от внутренних угроз ИБ» \\
    (2021)
    \item Второе место в рейтинге стажёров года в Касперском (2022)
\end{itemize}



\end{document}